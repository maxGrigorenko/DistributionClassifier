\documentclass[a4paper,12pt]{article}
\usepackage[utf8]{inputenc}
\usepackage[russian]{babel}
\usepackage{amsmath}
\usepackage{graphicx}
\usepackage{geometry}
\usepackage{color}
\usepackage{fancyhdr}
\usepackage{titlesec}
\usepackage{setspace}

\geometry{top=2cm,bottom=2cm,left=2cm,right=2cm}

% Настройка заголовков
\titleformat{\section}{\large\bfseries\color{blue}}{}{0em}{}[\titlerule]
\titleformat{\subsection}{\bfseries\color{blue}}{}{0em}{}

% Настройка шапки и подвала
\pagestyle{fancy}
\fancyhf{}
\fancyhead[L]{Классификация распределения с помощью случайных графов}
\fancyhead[R]{\thepage}
\fancyfoot[C]{\textit{Соколовский С.П., Григоренко М.Д.}}

% Настройка интервала между строками
\onehalfspacing

\title{\textbf{Классификация распределения с помощью случайных графов}} 
\author{Соколовский С.П., Григоренко М.Д.}
\date{Дата: \today}

\begin{document}

\maketitle

\section{Введение}
В этом разделе вы можете описать цель лабораторной работы, ее актуальность и основные задачи. Укажите, какие вопросы вы собираетесь рассмотреть.

\section{Теоретическая часть}
Здесь вы можете привести необходимые теоретические сведения, формулы и определения, которые будут полезны для понимания работы. Например, можно добавить формулы:

\begin{equation}
E = mc^2
\end{equation}

\section{Оборудование и материалы}
Перечислите все используемое оборудование и материалы, а также их характеристики. Например:
\begin{itemize}
    \item Оборудование 1: описание
    \item Оборудование 2: описание
    \item Материал 1: описание
\end{itemize}

\section{Методика проведения эксперимента}
Опишите шаги, которые вы предприняли для выполнения лабораторной работы. Укажите, какие методы и подходы вы использовали. 

\section{Результаты}
Представьте полученные результаты в виде таблиц и графиков. Например:

\begin{table}[h]
    \centering
    \begin{tabular}{|c|c|c|}
        \hline
        Параметр 1 & Параметр 2 & Результат \\
        \hline
        Значение 1 & Значение 2 & Результат 1 \\
        Значение 3 & Значение 4 & Результат 2 \\
        \hline
    \end{tabular}
    \caption{Таблица результатов}
\end{table}

\begin{figure}[h]
    \centering
    \includegraphics[width=0.7\textwidth]{example-image} % Замените на путь к вашему изображению
    \caption{График зависимости}
\end{figure}

\section{Обсуждение}
Проанализируйте полученные результаты, сравните их с теоретическими значениями, обсудите возможные источники ошибок. Укажите, что можно улучшить в методике.

\section{Заключение}
Подведите итоги вашей работы, сделайте выводы и рекомендации. Укажите, какие знания и навыки вы приобрели в ходе выполнения лабораторной работы.

\section{Список литературы}
\bibliographystyle{plain}
\bibliography{references}

\end{document}